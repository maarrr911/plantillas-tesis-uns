\documentclass[12pt]{../componentes/uns}
\usepackage[utf8]{inputenc}
\usepackage[spanish]{babel}
\usepackage{graphicx}
\usepackage{lipsum} % Para texto de relleno, eliminar en uso real

\title{Título del Proyecto de Investigación}
\author{Nombre del Autor}
\date{Lugar de Investigación, \today}

\begin{document}

% Portada
\maketitle
\newpage

% Tabla de contenido
\tableofcontents
\newpage

% Sección I: Datos Generales
\section*{I. Datos Generales}
\addcontentsline{toc}{section}{I. Datos Generales}

\subsection*{1.1 Título}
\addcontentsline{toc}{subsection}{1.1 Título}
Título del proyecto de investigación.

\subsection*{1.2 Autor}
\addcontentsline{toc}{subsection}{1.2 Autor}
Nombre del autor o autores.

\subsection*{1.3 Tipo de Investigación}
\addcontentsline{toc}{subsection}{1.3 Tipo de Investigación}
Especificar si es descriptiva, experimental, correlacional, etc.

\subsection*{1.4 Lugar de la Investigación}
\addcontentsline{toc}{subsection}{1.4 Lugar de la Investigación}
Especificar el lugar geográfico donde se desarrollará la investigación.

% Sección II: Plan de la Investigación
\section*{II. Plan de la Investigación}
\addcontentsline{toc}{section}{II. Plan de la Investigación}

\subsection*{2.1 Objeto de la Investigación}
\addcontentsline{toc}{subsection}{2.1 Objeto de la Investigación}
\subsubsection*{2.1.1 Realidad Genérica del Problema}
\addcontentsline{toc}{subsubsection}{2.1.1 Realidad Genérica del Problema}
\lipsum[1]

\subsubsection*{2.1.2 Características de la Realidad Específica}
\addcontentsline{toc}{subsubsection}{2.1.2 Características de la Realidad Específica}
\lipsum[2]

\subsection*{2.2 Formulación del Problema}
\addcontentsline{toc}{subsection}{2.2 Formulación del Problema}
\lipsum[3]

\subsection*{2.3 Antecedentes de la Investigación}
\addcontentsline{toc}{subsection}{2.3 Antecedentes de la Investigación}
\lipsum[4]

\subsection*{2.4 Justificación e Importancia}
\addcontentsline{toc}{subsection}{2.4 Justificación e Importancia}
\lipsum[5]

\subsection*{2.5 Limitaciones}
\addcontentsline{toc}{subsection}{2.5 Limitaciones}
\lipsum[6]

\subsection*{2.6 Hipótesis de la Investigación}
\addcontentsline{toc}{subsection}{2.6 Hipótesis de la Investigación}
\lipsum[7]

\subsection*{2.7 Variables}
\addcontentsline{toc}{subsection}{2.7 Variables}
\subsubsection*{2.7.1 Definición Conceptual}
\addcontentsline{toc}{subsubsection}{2.7.1 Definición Conceptual}
\lipsum[8]

\subsubsection*{2.7.2 Definición Operacional}
\addcontentsline{toc}{subsubsection}{2.7.2 Definición Operacional}
\lipsum[9]

\subsubsection*{2.7.3 Indicadores}
\addcontentsline{toc}{subsubsection}{2.7.3 Indicadores}
\lipsum[10]

\subsection*{2.8 Objetivos de la Investigación}
\addcontentsline{toc}{subsection}{2.8 Objetivos de la Investigación}
\subsubsection*{2.8.1 Objetivo General}
\addcontentsline{toc}{subsubsection}{2.8.1 Objetivo General}
\lipsum[11]

\subsubsection*{2.8.2 Objetivos Específicos}
\addcontentsline{toc}{subsubsection}{2.8.2 Objetivos Específicos}
\lipsum[12]

% Sección III: Fundamentación Teórica
\section*{III. Fundamentación Teórica}
\addcontentsline{toc}{section}{III. Fundamentación Teórica}

\subsection*{3.1 Marco Teórico}
\addcontentsline{toc}{subsection}{3.1 Marco Teórico}
\lipsum[13]

\subsection*{3.2 Marco Conceptual}
\addcontentsline{toc}{subsection}{3.2 Marco Conceptual}
\lipsum[14]

% Sección IV: Metodología de la Investigación
\section*{IV. Metodología de la Investigación}
\addcontentsline{toc}{section}{IV. Metodología de la Investigación}

\subsection*{4.1 Método(s) de la Investigación}
\addcontentsline{toc}{subsection}{4.1 Método(s) de la Investigación}
\lipsum[15]

\subsection*{4.2 Procedimiento de la Investigación}
\addcontentsline{toc}{subsection}{4.2 Procedimiento de la Investigación}
\lipsum[16]

\subsection*{4.3 Diseño}
\addcontentsline{toc}{subsection}{4.3 Diseño}
\lipsum[17]

\subsection*{4.4 Población y Muestra}
\addcontentsline{toc}{subsection}{4.4 Población y Muestra}
\lipsum[18]

\subsection*{4.5 Técnicas e Instrumentos de Recolección de Datos}
\addcontentsline{toc}{subsection}{4.5 Técnicas e Instrumentos de Recolección de Datos}
\lipsum[19]

\subsection*{4.6 Procedimiento de la Recolección de Datos}
\addcontentsline{toc}{subsection}{4.6 Procedimiento de la Recolección de Datos}
\lipsum[20]

\subsection*{4.7 Técnicas de Procedimiento y Análisis de los Resultados}
\addcontentsline{toc}{subsection}{4.7 Técnicas de Procedimiento y Análisis de los Resultados}
\lipsum[21]

% Sección V: Cronograma
\section*{V. Cronograma}
\addcontentsline{toc}{section}{V. Cronograma}
\lipsum[22]

% Sección VI: Recursos
\section*{VI. Recursos}
\addcontentsline{toc}{section}{VI. Recursos}

\subsection*{6.1 Bienes}
\addcontentsline{toc}{subsection}{6.1 Bienes}
\lipsum[23]

\subsection*{6.2 Servicios}
\addcontentsline{toc}{subsection}{6.2 Servicios}
\lipsum[24]

% Sección VII: Presupuesto
\section*{VII. Presupuesto}
\addcontentsline{toc}{section}{VII. Presupuesto}
\lipsum[25]

% Sección VIII: Financiamiento
\section*{VIII. Financiamiento}
\addcontentsline{toc}{section}{VIII. Financiamiento}
\lipsum[26]

% Sección IX: Bibliografía
\section*{IX. Bibliografía}
\addcontentsline{toc}{section}{IX. Bibliografía}
Las referencias deben numerarse según el estilo de redacción de la American Psychological Association (APA).

% Sección X: Anexos
\section*{X. Anexos}
\addcontentsline{toc}{section}{X. Anexos}
\lipsum[27]

\end{document}
