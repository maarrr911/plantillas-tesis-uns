\documentclass[12pt]{../componentes/uns}
\usepackage[utf8]{inputenc}
\usepackage[spanish]{babel}
\usepackage{graphicx}
\usepackage{setspace} % Para ajuste de espacio entre líneas
\usepackage{lipsum} % Para texto de relleno, eliminar en uso real

\title{Título de la Tesis Cualitativa}
\author{Nombre del Autor}
\date{Lugar de Investigación, \today}

\begin{document}

% Portada
\begin{titlepage}
    \centering
    \includegraphics[width=0.3\textwidth]{../componentes/logo_uns.png} \\
    \vspace{1cm}
    \textbf{\Large Universidad Nacional del Santa}\\
    \textbf{\large Escuela de Postgrado}\\
    Programa de Maestría en ............................................ \\
    \vspace{2cm}
    \textbf{\LARGE Título de la Tesis Cualitativa}\\
    \vspace{2cm}
    Tesis de Maestría en .........................................\\
    Autor(a): Nombres y Apellidos\\
    Asesor(es): Nombres y Apellidos\\
    \vfill
    Lugar y Año\\
    Registro N° ......................\\
\end{titlepage}

% Página en blanco
\newpage
\thispagestyle{empty}
\mbox{}

% Hoja de conformidad del asesor
\newpage
\section*{Hoja de conformidad del asesor}
Texto de conformidad.

% Hoja de aprobación del Jurado Evaluador
\newpage
\section*{Hoja de aprobación del Jurado Evaluador}
Texto de aprobación.

% Página de dedicatoria (opcional)
\newpage
\section*{Dedicatoria}
\lipsum[1] % Relleno de ejemplo

% Página de agradecimiento (opcional)
\newpage
\section*{Agradecimientos}
\lipsum[2] % Relleno de ejemplo

% Índice
\newpage
\tableofcontents
\newpage

% Lista de gráficos
\listoffigures
\newpage

% Lista de cuadros
\listoftables
\newpage

% RESUMEN
\section*{RESUMEN}
\addcontentsline{toc}{section}{RESUMEN}
\lipsum[3] % Relleno de ejemplo

% ABSTRACT
\section*{ABSTRACT}
\addcontentsline{toc}{section}{ABSTRACT}
\lipsum[4] % Relleno de ejemplo

% Introducción
\newpage
\section{INTRODUCCIÓN}
\addcontentsline{toc}{section}{INTRODUCCIÓN}
\lipsum[5]

% CAPÍTULO I: Problema de Investigación
\newpage
\section{CAPÍTULO I: PROBLEMA DE INVESTIGACIÓN}
\subsection{1.1 Presentación y delimitación del problema}
\lipsum[6]

\subsection{1.2 Objeto de estudio}
\lipsum[7]

\subsection{1.3 Formulación del problema (o Pregunta Orientadora)}
\lipsum[8]

\subsection{1.4 Objetivos ó Hipótesis}
\subsubsection{Generales}
\lipsum[9]

\subsubsection{Específicos}
\lipsum[10]

% CAPÍTULO II: Marco Teórico
\newpage
\section{CAPÍTULO II: MARCO TEÓRICO}
\subsection{2.1 Antecedentes}
\lipsum[11]

\subsection{2.2 Marco conceptual}
\lipsum[12]

\subsection{2.3 Marco referencial}
\lipsum[13]

\subsection{2.4 Definición de variables o conceptos principales}
\lipsum[14]

% CAPÍTULO III: Material y Métodos/Metodología de la Investigación
\newpage
\section{CAPÍTULO III: MATERIAL Y MÉTODOS/METODOLOGÍA DE LA INVESTIGACIÓN}
\subsection{3.1 Tipo de investigación}
\lipsum[15]

\subsection{3.2 Universo, incluye criterios de inclusión}
\lipsum[16]

\subsection{3.3 Muestra (o Sujetos participantes)}
\lipsum[17]

\subsection{3.4 Unidad de Muestreo/Análisis}
\lipsum[18]

\subsection{3.5 Instrumentos para la recolección de datos}
\lipsum[19]

\subsection{3.6 Procedimiento}
\lipsum[20]

\subsection{3.7 Control de calidad de los instrumentos}
\lipsum[21]

% CAPÍTULO IV: Resultados o Hallazgos
\newpage
\section{CAPÍTULO IV: RESULTADOS (O HALLAZGOS)}
\lipsum[22]

% CAPÍTULO V: Discusión
\newpage
\section{CAPÍTULO V: DISCUSIÓN}
\lipsum[23]

% CAPÍTULO VI: Conclusiones y Sugerencias
\newpage
\section{CAPÍTULO VI: CONCLUSIONES Y SUGERENCIAS}
\subsection{6.1 Conclusiones}
\lipsum[24]

\subsection{6.2 Recomendaciones}
\lipsum[25]

% Referencias Bibliográficas
\newpage
\section*{REFERENCIAS BIBLIOGRÁFICAS}
\addcontentsline{toc}{section}{REFERENCIAS BIBLIOGRÁFICAS}
Las referencias deben numerarse según el estilo de redacción de la American Psychological Association (APA).

% Anexos
\newpage
\section*{ANEXOS}
\addcontentsline{toc}{section}{ANEXOS}
\lipsum[26]

\end{document}
