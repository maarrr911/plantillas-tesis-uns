\documentclass[12pt]{../componentes/uns}
\usepackage[utf8]{inputenc}
\usepackage[spanish]{babel}
\usepackage{graphicx}
\usepackage{lipsum} % Para texto de relleno, eliminar en uso real

\title{Título del Proyecto de Investigación Cualitativa}
\author{Nombre del Autor}
\date{Lugar de Investigación, \today}

\begin{document}

% Portada
\maketitle

\newpage
\tableofcontents
\newpage

% Sección I: Datos Generales
\section*{I. Datos Generales}
\addcontentsline{toc}{section}{I. Datos Generales}

\subsection*{1. Título}
\addcontentsline{toc}{subsection}{1. Título}
Título del proyecto de investigación.

\subsection*{2. Autor}
\addcontentsline{toc}{subsection}{2. Autor}
Nombre del autor o autores.

\subsection*{3. Tipo de Investigación}
\addcontentsline{toc}{subsection}{3. Tipo de Investigación}
Especificar el tipo de investigación cualitativa.

\subsection*{4. Lugar de la Investigación}
\addcontentsline{toc}{subsection}{4. Lugar de la Investigación}
Especificar el lugar geográfico donde se desarrollará la investigación.

% Sección II: Plan de la Investigación
\section*{II. Plan de la Investigación}
\addcontentsline{toc}{section}{II. Plan de la Investigación}

\subsection*{2.1 Objeto de la Investigación}
\addcontentsline{toc}{subsection}{2.1 Objeto de la Investigación}
\subsubsection*{2.1.1 Realidad Genérica del Problema}
\addcontentsline{toc}{subsubsection}{2.1.1 Realidad Genérica del Problema}
\lipsum[1] % Relleno de texto

\subsubsection*{2.1.2 Características de la Realidad Específica}
\addcontentsline{toc}{subsubsection}{2.1.2 Características de la Realidad Específica}
\lipsum[2] % Relleno de texto

\subsection*{2.2 Formulación del Problema}
\addcontentsline{toc}{subsection}{2.2 Formulación del Problema}
\lipsum[3]

\subsection*{2.3 Antecedentes de la Investigación}
\addcontentsline{toc}{subsection}{2.3 Antecedentes de la Investigación}
\lipsum[4]

\subsection*{2.4 Justificación e Importancia}
\addcontentsline{toc}{subsection}{2.4 Justificación e Importancia}
\lipsum[5]

\subsection*{2.5 Limitaciones}
\addcontentsline{toc}{subsection}{2.5 Limitaciones}
\lipsum[6]

\subsection*{2.6 Objetivos de la Investigación}
\addcontentsline{toc}{subsection}{2.6 Objetivos de la Investigación}
\subsubsection*{2.6.1 Objetivo General}
\addcontentsline{toc}{subsubsection}{2.6.1 Objetivo General}
\lipsum[7]

\subsubsection*{2.6.2 Objetivos Específicos}
\addcontentsline{toc}{subsubsection}{2.6.2 Objetivos Específicos}
\lipsum[8]

% Sección III: Referencial Teórico – Empírico
\section*{III. Referencial Teórico – Empírico}
\addcontentsline{toc}{section}{III. Referencial Teórico – Empírico}
\lipsum[9]

% Sección IV: Referencial Metodológico
\section*{IV. Referencial Metodológico}
\addcontentsline{toc}{section}{IV. Referencial Metodológico}

\subsection*{4.1 Naturaleza y Método(s) de Investigación}
\addcontentsline{toc}{subsection}{4.1 Naturaleza y Método(s) de Investigación}
\lipsum[10]

\subsection*{4.2 Marco Contextual/Escenario del Estudio}
\addcontentsline{toc}{subsection}{4.2 Marco Contextual/Escenario del Estudio}
\lipsum[11]

\subsection*{4.3 Sujetos de Investigación (Muestra: Criterios de Selección)}
\addcontentsline{toc}{subsection}{4.3 Sujetos de Investigación (Muestra: Criterios de Selección)}
\lipsum[12]

\subsection*{4.4 Técnicas e Instrumentos}
\addcontentsline{toc}{subsection}{4.4 Técnicas e Instrumentos}
\lipsum[13]

\subsection*{4.5 Recojo de la Información}
\addcontentsline{toc}{subsection}{4.5 Recojo de la Información}
\lipsum[14]

\subsection*{4.6 Procesamiento de la Información}
\addcontentsline{toc}{subsection}{4.6 Procesamiento de la Información}
\lipsum[15]

\subsection*{4.7 Análisis de la Información}
\addcontentsline{toc}{subsection}{4.7 Análisis de la Información}
\lipsum[16]

\subsection*{4.8 Presentación de Resultados}
\addcontentsline{toc}{subsection}{4.8 Presentación de Resultados}
\lipsum[17]

% Sección V: Criterios Éticos y de Rigor
\section*{V. Criterios Éticos y de Rigor}
\addcontentsline{toc}{section}{V. Criterios Éticos y de Rigor}
\lipsum[18]

% Sección VI: Referencias Bibliográficas
\section*{VI. Referencias Bibliográficas}
\addcontentsline{toc}{section}{VI. Referencias Bibliográficas}
Las referencias deben numerarse según el estilo de redacción de la American Psychological Association (APA).

% Sección VII: Cronograma
\section*{VII. Cronograma}
\addcontentsline{toc}{section}{VII. Cronograma}
\lipsum[19]

% Sección VIII: Recursos
\section*{VIII. Recursos}
\addcontentsline{toc}{section}{VIII. Recursos}

\subsection*{8.1 Bienes}
\addcontentsline{toc}{subsection}{8.1 Bienes}
\lipsum[20]

\subsection*{8.2 Servicios}
\addcontentsline{toc}{subsection}{8.2 Servicios}
\lipsum[21]

% Sección IX: Presupuesto
\section*{IX. Presupuesto}
\addcontentsline{toc}{section}{IX. Presupuesto}
\lipsum[22]

% Sección X: Financiamiento
\section*{X. Financiamiento}
\addcontentsline{toc}{section}{X. Financiamiento}
\lipsum[23]

% Sección XI: Anexos
\section*{XI. Anexos}
\addcontentsline{toc}{section}{XI. Anexos}
\lipsum[24]

\end{document}
